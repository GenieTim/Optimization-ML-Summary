\section*{Introduction into Nonlinear Finite Element Analysis}

From continuum: $\div{\sigma} + b = 0, u_s = \bar{u} \text{ on } \partial\Omega_u, \sigma \cdot n = t = \bar{t} \text{ on } \partial\Omega_t$
to a discrete system: $K\hat{u}=f$ (linear FEM) and $K\Delta \hat{u} = \Delta f$ (nonlinear).
isoparametric concept: 
The geometry as well as the displacements field is approximated with the same shape functions.
$\hat{u}$: nodal displacements. $N$: matrix of interpolation functions. $B$: strain displacement matrix.
Mapping from isoparametric space to real space by Jacobi $J^{-1}$.

\textbf{4-Node bilinear element}
has shape/interpolation functions $N_1 = \sfrac{1}{4}(1-r)(1-s)$, $N_2 = \sfrac{1}{4}(1+r)(1-s)$, $N_3 = \sfrac{1}{4}(1+r)(1+s)$, $N_4 = \sfrac{1}{4}(1-r)(1+s)$, interpolation matrix $N=
\begin{bmatrix}
N_1 & 0 & N_2 & 0 & N_3 & 0 & N_4 & 0 \\
0 & N_1 & 0 & N_2 & 0 & N_3 & 0 & N_4
\end{bmatrix}$, 
nodal displacements $\hat{u} = \begin{bmatrix}
\hat{u}_x^{(1)} & \hat{u}_y^{(1)} & \hat{u}_x^{(2)} & \hat{u}_y^{(2)} &
\hat{u}_x^{(3)} & \hat{u}_y^{(3)} & \hat{u}_x^{(4)} & \hat{u}_y^{(4)}
\end{bmatrix}$ 
and strain interpolation matrix $B_i = \begin{bmatrix}
\sfrac{\partial N_i}{\partial x} & 0 \\
0 & \sfrac{\partial N_i}{\partial y} \\
\sfrac{\partial N_i}{\partial y} & \sfrac{\partial N_i}{\partial x} \\
\end{bmatrix}$, 
$J^{-1}\begin{bmatrix}
\sfrac{\partial N_i}{\partial r} \\
\sfrac{\partial N_i}{\partial s}
\end{bmatrix} =\begin{bmatrix}
\sfrac{\partial N_i}{\partial x} \\
\sfrac{\partial N_i}{\partial y}
\end{bmatrix}$, 
$B= \begin{bmatrix}
B_1 & B_2 & B_3 & B_4
\end{bmatrix}$.
Element force vectors are given by
$f_e^{int} = \int_{\Omega^e} B^T \sigma dV$, $f_e^{ext} = \int_{\Omega^e} N^T b  dV + \int_{\partial\Omega^e} N^T t  da$.
Using numerical integration, that is $f_e^{int} = \sum_{i=1}^{n_{\text{gaussp}}} w_i B_i ^T \sigma_i \det{J_i}$ and $f_e^{ext} = \sum_{i=1}^{n_{\text{gaussp}}} w_i N_i ^T b_i \det{J_i} + \sum_{i=1}^{n_{\text{gaussp}}} w_i^b N_i ^T t_i \sigma_i \det{J_i^b}$, respectively.
Assemble globally: $f^{int} = \bigcup_{e=1}^{n_{\text{elem}}} f_e^{int}$,  $f^{ext} = \bigcup_{e=1}^{n_{\text{elem}}} f_e^{ext}$ ($\bigcup$: finite element assembly operator).

For linear elasticity, we have the elasticity matrix $D^e$ related as $\sigma= D^e \epsilon = D^e B \hat{u}$, and the stiffness matrix $K ^e = \int_{\Omega^e} B^T D^e B dV$.

\textbf{Nonlinear FEM}
Material tensor $D$ is not constant. Iteration of Newton scheme requires solving $K \delta \hat{u}^{(k)} = r^{(k-1)}$ for $\delta \hat{u}^{(k)}$.
$r^{(k-1)}$ are the residual forces.
Update the global displacements as $\hat{u}_{n+1}^{(k)} = \hat{u}_{n+1}^{(k-1)} + \delta \hat{u}^{(k)}$.

\subsection*{Constrained Optimization in FEM}
\textbf{Contact Modelling (Imposition of Constraints)}
\textit{Penalty Method:}
The potential energy of a discretized system is given by $\Pi = \sfrac{1}{2} U^T K U - U^T R$ with $U\in\doubleR^n$: nodal displacements, $K\in\doubleR^{n\times n}$: global stiffness matrix and $R\in\doubleR^n$: external forces on nodes.
First term: elastic, second: potential (of external forces) energy.
Boundary conditions: $ZU=V$, $Z\in\doubleR^{m \times n}, V\in \doubleR^m$, $m$: Nr. of boundary conditions. 
Add the violation as a penalty term:
$\Pi = \sfrac{1}{2} U^T K U - U^T R + \sfrac{\alpha}{2}(ZU-V)^T(ZU-V)$.
Choose $\alpha \gg K_{?,?}$.
To minimize: $\partial\Pi = \partial U^T (KU -R+\alpha Z^T (ZU-V))=0$.
Has to be fulfilled for every $\partial U^T \Rightarrow U = (K+\alpha Z^T Z)^{-1} (R+\alpha Z^T V)$

\textit{Lagrange Multiplier Method:}
$\Pi = \sfrac{1}{2} U^T K U - U^T R + \lambda^T (ZU-V)$, and with $\partial \Pi = 0$, we get $\begin{bmatrix}
K & Z^T \\
Z & 0
\end{bmatrix} \begin{bmatrix}
U \\
\lambda
\end{bmatrix} = \begin{bmatrix}
R \\
V
\end{bmatrix}$.
Where $\lambda$ is a vector of $m$ Lagrange multipliers.
Disadvantages: Bigger bandwidth $\rightarrow$ loss of $K$ typical band structure. The system matrix is symmetric but not positive definite.