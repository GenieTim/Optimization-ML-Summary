\section*{Optimization based on Meta Modeling Techniques}
Metamodels tend to reduce the computational time significantly.

\textbf{Response Surface Method (RSM)}
Fit a surface to discrete points. Use the constructed function (meta-model) to solve the optimization (sub)problem.
Advantages: Tendency to approximate the global minimum. Avoids local outliers \& oscillations of the response thanks to smoothing. 
Optimization of RS leads to simple computations. 
Disadvantages: Efficient computations are restricted to a small number of variables (ca. 15).

\textbf{Error analysis}
Coefficient of determination:
$R^2 = \frac{\sum_{i=1}^{n_p} (\hat{y}_i-\bar{y})^2}{\sum_{i=1}^{n_p} (y_i-\bar{y})^2}$
with $n_p$: number of exp. points, $y_i$: actual response, $\hat{y}_i$: predicted response, $\bar{y}$: mean values of responses.
$0\leq R^2 \leq 1$. $R^2=1$ indicates perfect fit.
Small $R^2$: the region of interest is too large or too small.
Overfitting possible. 
$R^2$ represents the ability of the response surface to predict the variation of the true response.

\textbf{Successive Response Surface Method (SRSM)}
Adapt the subdesign space.

\textbf{Kriging}
High quality approximation, very flexible models. 

Meta-model $y=f(x)+Z(x)$, with $y(x)$: unknown interest, $f(x)$: known polynomial, regression coefficient $a$, $Z(x)$: stochastic component with mean zero and $\cov{[Z(x^i), Z(x^j)]} = \sigma^2 R[R(x^i, x^j)]$, with $R$: $n\times n$ correlation matrix (symmetric, unit diagonal) with $R(x^i, x^j)$ correlation between $x^i, x^j$.
E.g. exponential: $R=\prod_{k=1}^{m}\exp{-\theta_k \abs{d_k}}$ or Gaussian: $R=\prod_{k=1}^{m}\exp{-\theta_k \abs{d_k^2}}$ with $m$: number of variables, $d_k=x_k^i - x_k ^j$.
$a, \sigma, \theta$ are unknown and depend on each other. 
Iterative procedure required!
An appropriate combination of the regression model and the stochasitc model as well as the number and the distribution of the sampling points is of crucial importance.
In general, the computational time of the fitting process is larger than using RSM.

\textbf{Validation (Cross-Validation) of the Meta-model}
1. Remove sample $x_i$ from the data set.
2. Evaluate the model for the remaining $n_p - 1$ samples.
3. Evaluate the model on the removed sample $x_i \rightarrow \hat{y}_{-1}$.
4. Repeat the points 1. – 3. for each point of the data set.
5. Compare the computed values with the data points of the real supporting points.
$\epsilon_{RMS}=\sqrt{\frac{1}{n_p} \sum_{i=1}^{n_p} (y_i-\hat{y}_{-i})^2}$.


