\section*{Shape and Topology Optimization}
Discretize Geometry (CAGD) for the design model and mechanical model (FEM) for the analysis model to formulate the optimization problem. 

\textbf{SIMP-Approach for Continuous Material Topology Optimization}
SIMP = “Solid Isotropic Microstructure with Penalty for intermediate density”. 

\textbf{Analytical sensitivity analysis (computation of derivatives)}
Consider the linear FEM equations for the compliance problem:
$c(\rho_e) = u^T (\cup_{e=1}^{N}\rho_e^p K_e) u$,
$\min_{\rho_e} c(\rho_e)$, 
$\sum_{e=1}^N v_e \rho_e \leq V$, 
$0 < \rho_{\text{min}}\leq \rho_e \leq 1$, 
$e=1,...N$ (e: element, V: volume).
The derivative of the compliance is of the particularly simple form (at the element level):
$\sfrac{\partial c}{\partial \rho_e} = -p \rho_e^{p-1} u^T K_e u$.
Computation of the derivative of the objective function with respect to the design variables is required, if a gradient-based optimization method is applied!

\textbf{Filtering the sensitivities (numerical stabilization)}
Filtering of the sensitivity information is an efficient way to ensure mesh independency. 
This means: modifying the design sensitivity of a specific element, based on the weighted average of the
element sensitivities in a fixed neighborhood.
$\rightarrow$ no extra constraints need to be considered (very simple to implement)
$\sfrac{\hat{\partial c}}{\partial \rho_k} = \sfrac{1}{\rho_k \sum_{i=1}^N \hat{H_i}} \sum_{i=1}^N \hat{H_i} \rho_i \sfrac{\partial c}{\partial \rho_i}$.

The mesh independent factor $\hat{H_i}$ is written as
$\hat{H_i} = r_{\text{min}} - \text{dist}(k,i), \{i \in N | \text{dist}(k,i) \leq r_{\text{min}}\}$, 
$k = 1,...,N$ with $\text{dist}(k,i)$: distance between the center of element $k$ and element $i$.

\subsection*{Shape Optimization}
Determine the optimal shape of edges and surfaces. No change in topology.
Transformation of the variational problem into a parameter optimization problem
$\rightarrow$ Optimization variables $s$: free, discrete parameters which describe the shape of the structure.

\textbf{FE-node based approach}:
FE-nodes are directly defined as design variables.
Most freedom of shape change, no time-consuming shape parametrization process is required.
Either without filtering/smoothing techniques: lacks a length scale control that is necessary to ensure a well-posed shape optimization problem and avoid numerical instability, 
or with: motivated by the success of filtering techniques that impose minimum length scales in topology optimization.

\textbf{Computer Aided Geometric Design (CAGD) – based approach}
Design Elements (Splines, B-Splines, Bézier-Splines,....). Much less design variable than FE-node based.
Approximate geometry of the structure piecewise using so-called design elements (curves, surface, or volume elements).
Optimization variable: Position of the control points of the design elements. 
In general, the FE-mesh is much finer than the discretization with design elements. 
Because of the strict separation between the geometry model and analysis model, numerical instabilities are avoided.
Optimized shape is available in CAD format. 
Disadvantages:
Couplings between the CAGD model and the FE-model have to be supported by the program.
Problems due to re-meshing or definition of the new CAD model can occur.

